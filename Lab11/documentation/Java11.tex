\documentclass{article}
\usepackage[english,russian]{babel}
\usepackage[utf8]{inputenc}
\usepackage{indentfirst}
\usepackage{graphicx}
\usepackage{float}
\usepackage[margin=2cm]{geometry}

\begin{document}
\begin{titlepage}
	\begin{center}
    	ГУАП
    	\vspace{0.25cm}

    	КАФЕДРА №51
	\end{center}

    \begin{flushleft}

    	ОТЧЕТ

    	ЗАЩИЩЕН С ОЦЕНКОЙ

		ПРЕПОДАВАТЕЛЬ 


    	\vspace{0.5cm} 

		$\rule{5cm}{0.15mm}$ \hfill $\rule{2.2cm}{0.15mm}$  \hfill $\rule{3.25cm}{0.15mm}$

		должность, уч. степень, звание \hfill подпись, дата \hfill инициалы, фамилия
    \end{flushleft}
    
 	
    \hspace{2cm}

	\begin{center}
    	ОТЧЕТ ПО ЛАБОРАТОРНОЙ РАБОТЕ №11


    	\vspace{1cm}

    	ЧАТ ДЛЯ ДВУХ ПОЛЬЗОВАТЕЛЕЙ


    	\vspace{1cm}

    	по курсу: ОСНОВЫ ПРОГРАММИРОВАНИЯ {\MakeUppercase{\romannumeral 2}}
    \end{center}

    \vspace{3cm}

    \begin{flushleft}
    	РАБОТУ ВЫПОЛНИЛ

    	СТУДЕНТ ГР. № 5511 \hfill $\rule{2.2cm}{0.15mm}$  \hfill $\rule{3.25cm}{0.15mm}$

    	\hspace{7.8cm} подпись, дата \hfill инициалы, фамилия
    \end{flushleft}

	\vspace{5cm}   
	\begin{center}
 		Санкт-Петербург 2017
	\end{center}
\end{titlepage}

\section{Задание}
Написать текстовый чат для двух пользователей на сокетах. Чат должен быть реализован по принципу клиент-сервер. Один пользователь находится на сервере, второй --- на клиенте. Адреса и порты задаются через командную строку: клиенту --- куда соединяться, серверу --- на каком порту слушать. При старте программы выводится текстовое приглашение, в котором можно ввести одну из следующих команд:
\begin{itemize}
	\item задать имя пользователя (@name Vasya)

	\item послать текстовое сообщение (Hello)

	\item выход (@quit)
\end{itemize}

Принятые сообщения автоматически выводятся на экран. Программа работает по протоколу UDP.

\section{Дополнительное задание}
Реализовать интерфейс для чата при помощи Swing.

\section{Реализация}
Программа реализована тремя частями --- клиент, сервер и графический интерфейс.

\subsection{Клиент}
Клиентская часть реализована в виде класса ClientUDP. Получает на вход адрес сервера, куда подключиться, и номер порта. Имплементирует интерфейс Runnable и пользовательский интерфейс Chat, необходимый для работы с графическим интерфейсом. В поле хранит сокет, обеспечивающий связь с сервером. В методе run() в отдельном потоке принимаются и обрабатываются пакеты с сервера. 

\subsection{Сервер}
Реализован в виде класса ServerUDP. Получает на вход аргументом командной строки порт, на котором слушать. Имплементирует интерфейс Runnable и пользовательский интерфейс Chat, необходимый для работы с графическим интерфейсом. В поле хранит сокет, обеспечивающий связь с клиентом. В методе run() в отдельном потоке принимаются и обрабатываются пакеты от клиента.

\subsection{Графический интерфейс}
Реализован классом Interface при помощи Swing. Имеет поле Сhat, где хранится соответствующий ему объект клиента или сервера. Отвечает за отображение пришедших сообщений и передачи ввода пользователя клиенту или серверу.

\section{Инструкция}
Запуск происходит из командной строки.

\subsection{Запуск сервера}
Порт передается аргументом командной строки. Пример вызова --- ServerUDP.java 5000. При успешном запуске открывается окно с чатом и строкой ввода. После присоединения клиента можно приступать к использованию. Отправка сообщения производится вводом в строку и нажатием Enter.

Можно настроить имя, которое будет отображаться у собеседника, используя команду @name. Пример использования - @name Alice.

Для выхода из программы можно нажать на крестик в углу окна или ввести команду @end

\subsection{Запуск клиента}
Адрес сервера и порт передается аргументом командной строки. Пример вызова --- ClientUDP.java localhost 5000, где localhost --- адрес локальной машины. При успешном запуске открывается окно с чатом и строкой ввода.. Отправка сообщения производится вводом в строку и нажатием Enter.

Можно настроить имя, которое будет отображаться у собеседника, используя команду @name. Пример использования - @name Alice.

Для выхода из программы можно нажать на крестик в углу окна или ввести команду @end

\section{Тестирование}

\subsection{Пример запуска программы}
\begin{figure}[H]
	\begin{flushleft}
		\centerline{\includegraphics[scale=0.4]{chatexmp.jpg}}
		\caption{Пример запуска сервера с аргументами командной строки localhost 5000}
	\end{flushleft}
\end{figure}
\vspace{3cm}

\subsection{Пример работы программы}
\begin{figure}[H]
	\begin{flushleft}
		\centerline{\includegraphics[scale=0.4]{chattingexmp.jpg}}
		\caption{Пример пересылки сообщений между клиентом и сервером, а так же использования команды @name}
	\end{flushleft}
\end{figure}

\end{document}
