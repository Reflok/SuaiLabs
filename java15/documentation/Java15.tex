\documentclass{article}
\usepackage[english,russian]{babel}
\usepackage[utf8]{inputenc}
\usepackage{indentfirst}
\usepackage{graphicx}
\usepackage{float}
\usepackage[margin=2cm]{geometry}

\begin{document}
\begin{titlepage}
	\begin{center}
    	ГУАП
    	\vspace{0.25cm}

    	КАФЕДРА №51
	\end{center}

    \begin{flushleft}

    	ОТЧЕТ

    	ЗАЩИЩЕН С ОЦЕНКОЙ

		ПРЕПОДАВАТЕЛЬ 


    	\vspace{0.5cm} 

		$\rule{5cm}{0.15mm}$ \hfill $\rule{2.2cm}{0.15mm}$  \hfill $\rule{3.25cm}{0.15mm}$

		должность, уч. степень, звание \hfill подпись, дата \hfill инициалы, фамилия
    \end{flushleft}
    
 	
    \hspace{2cm}

	\begin{center}
    	ОТЧЕТ ПО ЛАБОРАТОРНОЙ РАБОТЕ №15


    	\vspace{1cm}

    	HttpSession. CSS


    	\vspace{1cm}

    	по курсу: ОСНОВЫ ПРОГРАММИРОВАНИЯ {\MakeUppercase{\romannumeral 2}}
    \end{center}

    \vspace{3cm}

    \begin{flushleft}
    	РАБОТУ ВЫПОЛНИЛ

    	СТУДЕНТ ГР. № 5511 \hfill $\rule{2.2cm}{0.15mm}$  \hfill $\rule{3.25cm}{0.15mm}$

    	\hspace{7.8cm} подпись, дата \hfill инициалы, фамилия
    \end{flushleft}

	\vspace{5cm}   
	\begin{center}
 		Санкт-Петербург 2017
	\end{center}
\end{titlepage}

\section{Задание}
Реализовать сервлет для организации доски объявлений. Объявление содержит текст и заголовок (время размещения и имя пользователя).

Для того, чтобы разместить, необходимо ввести логин и пароль (пройти аутентификацию). При старте сервлет загружает базу пользователей и их паролей из текстового файла. Просматривать объявления можно без аутентификации (ввода логина и пароля).

На главной странице находится ссылка “войти в систему” и показывается список объявлений. После входа в систему добавляются ссылки: “выйти из системы”, “добавить объявление”. 
Для аутентификации необходимо использовать класс HttpSession
Для перенаправления пользователя на другую страницу и включения в  страницу готов кусков html кода можно воспользоваться классом RequestDispatcher
Протестировать, что при очистке cookie в браузере, пользователь выходит из системы.

Между перезагрузками сервера (рестарт Tomcat) список объявлений можно не сохранять.

Дополнительные требования (CSS):


\begin{enumerate}
	\item Оформление страницы должны быть задано в отдельном файле style.css
	\item Должны быть созданы три стиля:

	\begin{enumerate}
	\item Заголовк объявления 
	\item Текст объявления
	\item Ссылка с командой  (“войти в систему”, “выйти из системы”, “добавить объявления”)
	\end{enumerate}
	\item В работе должно быть два варианта оформления (переключение --- переименованием файлов на сервере)
\end{enumerate}

\section{Дополнительное задание}
Добавить ссылку "Ответить на объявление" для кажого объявления. Ответы выводятся под объявлением со смещением вправо. Отвечать на ответ нельзя. Отвечать могут только вошедшие пользователи.

\section{Реализация}
Программа реализована с использованием контейнера сервлетов Apache Tomcat.

Главный класс программы наследуется от класса HttpServlet и в методе doGet() в зависимости от запроса генерирует соответствующий ответ в виде HTML файла.
Свзязь сервлета с URL определена в файле web.xml.
Информация о зарегистрированных хранится в текстовом файле. Данные загружаются из файла в методе init() сервлета и сохраняются в него в методе destroy().
Для вывода форм и списков используется класс RequestDispatcher и его метод include(). 
Используются куки, так что дважды вводить логин и пароль не требуется.

\section{Инструкция}
После запуска сервера при посещении страницы localhost:8080/java15/main перед пользователем откроется форма для ввода логина или пароля, ссылка на просмотр объявлений и ссылка на страницу регистрации. После ввода корректных логина и пароля, или если вход был выполнен ранее, пользователь переходит на страницу с объявлениями. Там он может отставить новое объявление или ответить на уже существующее.

\section{Тестирование}

\subsection{Главная страница}
\begin{figure}[H]
	\begin{flushleft}
		\centerline{\includegraphics[scale=0.6]{login.jpg}}
		\caption{Ввод логина и пароля}
	\end{flushleft}
\end{figure}

\subsection{Регистрация}
\begin{figure}[H]
	\begin{flushleft}
		\centerline{\includegraphics[scale=0.6]{register.jpg}}
		\caption{Ввод логина и пароля для регистрации}
	\end{flushleft}
\end{figure}

\subsection{Создание нового объявления}
\begin{figure}[H]
	\begin{flushleft}
		\centerline{\includegraphics[scale=0.6]{newpost.jpg}}
		\caption{Форма для создания объявления}
	\end{flushleft}
\end{figure}

\begin{figure}[H]
	\begin{flushleft}
		\centerline{\includegraphics[scale=0.6]{posted.jpg}}
		\caption{Новое объявление появилось в списке}
	\end{flushleft}
\end{figure}

\subsection{Ответ на объявление}
\begin{figure}[H]
	\begin{flushleft}
		\centerline{\includegraphics[scale=0.6]{response.jpg}}
		\caption{Форма для создания ответа на объявление}
	\end{flushleft}
\end{figure}

\begin{figure}[H]
	\begin{flushleft}
		\centerline{\includegraphics[scale=0.6]{responded.jpg}}
		\caption{Ответ появился под объявлением}
	\end{flushleft}
\end{figure}

\subsection{Проверка очистки cookie}
\begin{figure}[H]
	\begin{flushleft}
		\centerline{\includegraphics[scale=0.6]{cookie.jpg}}
		\caption{После очистки cookie снова требуется ввести логин и пароль}
	\end{flushleft}
\end{figure}

\end{document}
