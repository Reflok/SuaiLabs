\documentclass{article}
\usepackage[english,russian]{babel}
\usepackage[utf8]{inputenc}
\usepackage{indentfirst}
\usepackage{graphicx}
\usepackage{float}
\usepackage[margin=2cm]{geometry}

\begin{document}
\begin{titlepage}
	\begin{center}
    	ГУАП
    	\vspace{0.25cm}

    	КАФЕДРА №51
	\end{center}

    \begin{flushleft}

    	ОТЧЕТ

    	ЗАЩИЩЕН С ОЦЕНКОЙ

		ПРЕПОДАВАТЕЛЬ 


    	\vspace{0.5cm} 

		$\rule{5cm}{0.15mm}$ \hfill $\rule{2.2cm}{0.15mm}$  \hfill $\rule{3.25cm}{0.15mm}$

		должность, уч. степень, звание \hfill подпись, дата \hfill инициалы, фамилия
    \end{flushleft}
    
 	
    \hspace{2cm}

	\begin{center}
    	ОТЧЕТ ПО ЛАБОРАТОРНОЙ РАБОТЕ №12


    	\vspace{1cm}

    	МНОГОПОЛЬЗОВАТЕЛЬСКИЙ ЧАТ


    	\vspace{1cm}

    	по курсу: ОСНОВЫ ПРОГРАММИРОВАНИЯ {\MakeUppercase{\romannumeral 2}}
    \end{center}

    \vspace{3cm}

    \begin{flushleft}
    	РАБОТУ ВЫПОЛНИЛ

    	СТУДЕНТ ГР. № 5511 \hfill $\rule{2.2cm}{0.15mm}$  \hfill $\rule{3.25cm}{0.15mm}$

    	\hspace{7.8cm} подпись, дата \hfill инициалы, фамилия
    \end{flushleft}

	\vspace{5cm}   
	\begin{center}
 		Санкт-Петербург 2017
	\end{center}
\end{titlepage}

\section{Задание}
Написать текстовый многопользовательский чат. 

\begin{enumerate}
	\item Пользователь управляет клиентом. На сервере пользователя нет. Сервер занимается пересылкой сообщений между клиентами.
	\item По умолчанию сообщение посылается всем участникам чата.
	\item Есть команда послать сообщение конкретному пользователю (@senduser Vasya).
\end{enumerate}

Программа работает по протоколу TCP.

\section{Дополнительное задание}
Совместное рисование в клетке 3х3. Пример команды "@draw 1 2 X". В ячейку с адресом в первом столбце и второй строке добавляется Х. Сетка отрисовывает после каждого изменения.

\section{Реализация}
Программа реализована тремя частями --- клиент, сервер и графический интерфейс

\subsection{Сервер}
Реализован в виде класса Server. Получает на вход аргументом командной строки порт, на котором слушать. Имплементирует интерфейс Runnable. В поле хранит сокет типа ServerSocket, в методе run() в отдельном потокое обрабатывающий новые соединения.

Для работы с отдельным клиентом написан внутренний класс ClientHandeler, так же имплементирующий интерфейс Runnable. При каждом новом соединений создается объект этого класса. В методе run() в отдельном потоке обрабатываются сообщения приходящие от клиента.

\subsection{Клиент}
Клиентская часть реализована в виде класса Client. Получает на вход адрес сервера, куда подключиться, и номер порта. Имплементирует интерфейс Runnable и пользовательский интерфейс Chat, необходимый для работы с графическим интерфейсом. В поле хранит сокет, обеспечивающий связь с сервером. В методе run() в отдельном потоке принимаются и обрабатываются сообщения сервера. 

Связь приходящей из сокета информации и отображения с помощью класса Interface обеспечивает внутренний класс InputHandeler. Ввод пользователя обрабатывает  и посылкает классу Client для отправки по сокетукласс Interface.

\subsection{Графический интерфейс}
Реализован классом Interface при помощи Swing. Имеет поле Client.InputHandeler, где хранится соответствующий ему объект клиента. Отвечает за отображение пришедших сообщений и передачи ввода пользователя клиенту для отправки.

\section{Инструкция}
Запуск происходит из командной строки.

\subsection{Запуск сервера}
Порт передается аргументом командной строки. Пример вызова --- Server.java 5000. 

Для выхода из программы надо ввести команду @stop.

\subsection{Запуск клиента}
Адрес сервера и порт передается аргументом командной строки. Пример вызова --- Client.java localhost 5000, где localhost --- адрес локальной машины. При успешном запуске открывается окно с чатом и строкой ввода и окно с клеткой 3 на 3 для рисования. Отправка сообщения производится вводом в строку и нажатием Enter.

Для рисования в клетке можно использовать команду @draw. Пример использования --- "@draw 1 1 X". Ставит символ X в центральную клетку.

Можно настроить имя, которое будет отображаться у собеседника, используя команду @name. Пример использования --- @name Alice.

Можно отправить приватное сообщение одному из пользователей. Пример использования --- "@Alice private message".

Для выхода из программы можно нажать на крестик в углу окна или ввести команду @stop.

\section{Тестирование}

\subsection{Пример запуска сервера}
\begin{figure}[H]
	\begin{flushleft}
		\centerline{\includegraphics[scale=0.6]{server.jpg}}
		\caption{Пример запуска и работы сервера с аргументом командной строки 5000}
	\end{flushleft}
\end{figure}
\vspace{3cm}

\subsection{Пример запуска клиента}
\begin{figure}[H]
	\begin{flushleft}
		\centerline{\includegraphics[scale=0.5]{client.jpg}}
		\caption{Пример запуска клиента с аргументами командной строки localhost 5000}
	\end{flushleft}
\end{figure}

\subsection{Пример общения клиентов}
\begin{figure}[H]
	\begin{flushleft}
		\centerline{\includegraphics[scale=0.5]{clientchatting.jpg}}
		\caption{Пересылка сообщений между тремя клиентами, с установкой имени и отправкой приватного сообщения}
	\end{flushleft}
\end{figure}

\subsection{Пример рисования}
\begin{figure}[H]
	\begin{flushleft}
		\centerline{\includegraphics[scale=0.4]{drawing.jpg}}
		\caption{Совместное рисование, все нарисованные элементы отображаются у всех пользователей}
	\end{flushleft}
\end{figure}

\end{document}
